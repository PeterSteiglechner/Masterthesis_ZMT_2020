% !TEX root = presentation_29Jun.tex

%\begin{frame}{Agent Based Modelling}
%\end{frame}
%
%\begin{frame}{Modelling Societies with ABM}
%
%%\begin{block}{\citet{Bookstaber2019}}
%%	\TODO
%%\end{block}
%\end{frame}


\begin{frame}{What is Agent-Based Modeling (ABM)}
%\begin{itemize}
%	\item Discrete units (Agents) in an environment with specific traits
%	\item Asynchronous and autonomous update of agents:
%	%\item In each update: 
%	\begin{itemize}
%		\item interaction based on rules and heuristics \\(often non-linear, heterogeneous, stochastic, adaptive, path-dependent)
%		\item autonomous decisions of agents
%	\end{itemize}
%	\item Aggregate variables are (1) outcomes of and (2) contexts for the agents decisions and actions \citep{Kohler2000}
%\end{itemize}



\begin{algorithm}[H]
	\textbf{Initialise} Environment  \\
	\textbf{Initialise} discrete Agents with specific traits in some location within the environment\\
	\For{$t \ \in\  \{t_{\rm start}, t_{\rm end}\}$}{
		\For{agent (in asynchronous order)}{
			\textbf{Interaction} between agent and environment based on rules and heuristics\\
			\textbf{Update} agent according to autonomous decisions
		}
		\textbf{Calculate} aggregate variables of environment and agents
	}
	\caption{General Structure}
\end{algorithm}
\end{frame}



\begin{frame}{Advantages of ABM with respect to ODE models}


% 	\begin{block}{\citet{Epstein1996}}%, as cited in \citet{Bonabeau2002}}
% 		 \begin{framed}
%	On Agent-Based Modelling:
%	"Perhaps one day people will interpret the question, ‘Can you explain [an observed social phenomenon]?’ as asking `Can you grow it?’"
%\end{framed}
%	 	\end{block}
 	 	%, as cited in \citet{Bonabeau2002}}
 	 	\begin{framed}
 	 		 \citet{Epstein1996}, \citet{Bonabeau2002} on Agent-Based Modelling:
 			\begin{center}
 				\textit{"Perhaps one day people will interpret the question, ‘Can you explain [an observed social phenomenon]?’ as asking `Can you grow it?’"
} 			\end{center}
 		\end{framed}

	From macroscopic, top-down description \ra to microscopic, bottom-up description
	\begin{enumerate}
		\item ABM allows for rules and heuristics to be 
		\begin{itemize}
			\item non-linear, discontinuous
			\item heterogeneous,
			\item stochastic, 
			\item adaptive \newline
			%\item path-dependent 
			\ra non-ergodicity \& computational irreducibility
		\end{itemize}
		\pause\item \ra Emergent phenomena
		%\item Non Ergodicity
		%\pause\item \ra Computational irreducibility
		\pause\item Decisions include stochasticity
		\pause\item Natural implementation of heterogeneous space dependency
	\end{enumerate}
\end{frame}


%\begin{frame}{Application of ABMs}
%\begin{itemize}
%	\item Typically: Problems related to cognitive psychology, game theory, and complexity science \citep{Bousquet2004}
%	\item Also, Socio-ecological Systems \citep{Muller-Hansen2017} \ra Human-Resource Interaction
%	\begin{itemize}
%		\item \ldots
%		\item Collapse of Ancient Societies
%		\begin{itemize}
%			\item Anasazi in Arizona, US \citet{Axtell2002}
%			\item Maya \citet{Heckbert2013}
%			\item Now: Easter Island
%		\end{itemize}
%	\end{itemize}
%\end{itemize}
%\end{frame}
