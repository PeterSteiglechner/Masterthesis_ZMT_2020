% !TEX root=frame_thesis.tex

The history of Easter Island, with its cultural and ecological mysteries, has attracted the interests of archaeologists, anthropologists, ecologists, and econo\-mists alike. Despite the great scientific efforts, uncertainties in the available archaeological and palynological data leave a number of critical issues unsolved and open to debate. The maximum size reached by the human population before the arrival of Europeans and the temporal dynamics of deforestation are some of the aspects still fraught with controversies. By providing a quantitative workbench for testing hypotheses and scenarios, mathematical models are a valuable complement to the observational-based approaches generally used to reconstruct the history of the island. Previous modelling studies, however, have shown a number of shortcomings in the case of Easter Island, especially when they take no account of the stochastic nature of population growth in a temporally and spatially varying environment. Here, I present a new stochastic, Agent-Based Model characterised by (1) realistic physical geography of the island and other environmental constraints (2) individual agent decision-making processes, (3) non-ergodicity of agent behaviour and environment, and (4) randomised agent-environment interactions. I use the model and the best available data to determine plausible spatial and temporal patterns of deforestation and other socioecological features of Easter Island prior to the European contact. 
I further identify some non-trivial connections between microscopic decisions or constraints (like local confinement of agents' actions or their adaptation strategy to environmental degradation) and macroscopic behaviour of the system that can not easily be neglected in a discussion about the history of Easter Island before European contact.


%		En Rumsligt Explicit Agentbaserad Modell av Interaktionen mellan Människor och Resurser på Påskön

%Påsköns historia har, med dess kulturella och ekologiska mysterier, väckt intressen hos arkeologer, antropologer, ekologer och ekonomer. 
%Trots de stora vetenskapliga ansträngningarna lämnar osäkerheten i de tillgängliga arkeologiska och palynologiska data ett antal kritiska frågor olösta och öppna för debatt. 
%Den maximala storleken som den mänskliga befolkningen nådde före européernas ankomst, och avskogningens tidsmässiga dynamik, är några av de aspekter som fortfarande är fyllda med kontroverser. 
%Genom att tillhandahålla en kvantitativ arbetsbänk för att testa hypoteser och scenarier är matematiska modeller ett värdefullt komplement till de observationsbaserade metoder som vanligtvis används för att rekonstruera öns historia. 
%Tidigare modelleringsstudier har emellertid visat ett antal brister i fallet med Påskön, särskilt när de inte tar hänsyn till den stokastiska karaktären av befolkningsökningen i en tillfällig och rumsligt varierande miljö. 
%Här presenters en ny stokastisk, agentbaserad modell som kännetecknas av (1) realistisk fysisk geografi av ön och andra miljömässiga begränsningar, (2) individuella beslutsprocesser av agenter, (3) icke-ergodicitet av agentens beteende och miljö och (4) randomiserade agent-miljöinteraktioner. 
%Modellen används tillsammans med de bästa tillgängliga data för att bestämma rimliga rumsliga och temporära mönster av avskogning och andra socioekologiska egenskaper på Påskön före européers ankoms. Vidare identifieras några icke-triviala förbindelser mellan mikroskopiska beslut eller begränsningar (till exempel lokal inneslutning av agentens handlingar eller deras anpassningsstrategi till miljöförstöring) och makroskopiskt beteende hos systemet som inte lätt kan försummas i en diskussion om påsköns historia före europeisk kontakt.