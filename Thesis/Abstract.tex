% !TEX root=frame_thesis.tex

The history of Easter Island, with its cultural and ecological mysteries, has attracted the interests of archaeologists, anthropologists, ecologists, and econo\-mists alike. Despite the great scientific efforts, uncertainties in the available archaeological and palynological data leave a number of critical issues unsolved and open to debate. The maximum size reached by the human population before the arrival of Europeans and the temporal dynamics of deforestation are some of the aspects still fraught with controversies. By providing a quantitative workbench for testing hypotheses and scenarios, mathematical models are a valuable complement to the observational-based approaches generally used to reconstruct the history of the island. Previous modelling studies, however, have shown a number of shortcomings in the case of Easter Island, especially when they take no account of the stochastic nature of population growth in a temporally and spatially varying environment. Here, I present a new stochastic, Agent-Based Model characterised by (1) realistic physical geography of the island and other environmental constraints (2) individual agent decision-making processes, (3) non-ergodicity of agent behaviour and environment, and (4) randomised agent-environment interactions. I use the model and the best available data to determine plausible spatial and temporal patterns of deforestation and other socioecological features of Easter Island prior to the European contact.% and provide a general proof of concept of the model.
I further identify some non-trivial connections between microscopic decisions or constraints (like local confinement of agents' actions or their adaption strategy to environmental degradation) and macroscopic behaviour of the system that can not easily be neglected in a discussion about the history of Easter Island.