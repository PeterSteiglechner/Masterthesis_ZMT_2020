% !TEX root=frame_thesis.tex
\chapter{Results}

\section{Standard Run}
 \begin{itemize}
 	\item Fig 1: Aggregate Dynamics (population, trees, mean penalties and happiness, fires, excess mortality/fertility). The plot that Esteban sketched during last Zoom Meeting [called statistics plot from now on]. All statistics plots are mean ensemble runs of 10 different seeds.
 	\item Comparison of the spatial deforestation pattern with \citet{Rull2020}'s Figure, which we sent to Valenti.
 \end{itemize}

\section{Different theories}
Statistics Plots for a
\begin{itemize}
	\item Run with low N fixation (instead of high)
	\item Run with tree regeneration and high N fixation
	\item Run with tree regeneration and low N fixation
\end{itemize}

\section{Different Tree Preference Functions, i.e.\ Agent Adaption}
\begin{itemize}
	\item Runs with $f_{\rm Tree \ Pref}$ delayed, careful, logistic.
\end{itemize}
I don't know yet if this makes any difference to the standard run and what difference. So we'll see.

\section{A less resilient society}
Run with a larger shape parameter of $g(H_{\rm i})(t)$.
I don't know yet if this makes any difference to the standard run and what difference. So we'll see.
I assume that the turning point of growing population to decreasing population size happens earlier. Maybe I'll just show the statistics plot.


\section{Senstivity Analysis of some uncertain parameters}
$r_{\rm T}$, $r_{\rm F}$,
$T_{\rm Req, pP}$

\section{Three different decision making processes}
\begin{itemize}
	\item Standard setting, $\gamma=20$, $\alpha=(0.2,0.2,0.2,0.2,0.2)$.
	\item `Only resource availability matters for moving decision' setting, $\gamma=20$, $\alpha=(0, 0, 0, 0.5,0.5)$.
	\item Hopping agents, that move to a random spot with uniform probability over the island, $\gamma=0$, $\alpha=$ doesn't matter.
\end{itemize}
Results not yet known. I guess I'll describe qualitatively what happens to the spatial patterns. 
If the aggregate dynamics change, this would of course be a big, big result and I would focus on that.


\section{Perhaps, if I have time: Regional Dynamics}
Adjust the statistics plot of total population dynamics to looking at a few hand-defined regions and the plot the aggregate dynamics of them separately. 

\section{Fires}
Plot the distribution of fires on the map over time. 
I imagine a map in which the color determines the timing of fires in each cell. 
I'm not sure yet how to do this exactly since fires in any cell occur at multiple times. Maybe I'll take the first occurence. We'll see.



\chapter{Discussion and Conclusion}
