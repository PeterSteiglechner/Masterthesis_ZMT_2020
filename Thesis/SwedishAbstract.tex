%		En Rumsligt Explicit Agentbaserad Modell av Interaktionen mellan Människor och Resurser på Påskön

Påsköns historia har, med dess kulturella och ekologiska mysterier, väckt intressen hos arkeologer, antropologer, ekologer och ekonomer. 
Trots de stora vetenskapliga ansträngningarna lämnar osäkerheten i de tillgängliga arkeologiska och palynologiska data ett antal kritiska frågor olösta och öppna för debatt. 
Den maximala storleken som den mänskliga befolkningen nådde före européernas ankomst, och avskogningens tidsmässiga dynamik, är några av de aspekter som fortfarande är fyllda med kontroverser. 
Genom att tillhandahålla en kvantitativ arbetsbänk för att testa hypoteser och scenarier är matematiska modeller ett värdefullt komplement till de observationsbaserade metoder som vanligtvis används för att rekonstruera öns historia. 
Tidigare modelleringsstudier har emellertid visat ett antal brister i fallet med Påskön, särskilt när de inte tar hänsyn till den stokastiska karaktären av befolkningsökningen i en tillfällig och rumsligt varierande miljö. 
Här presenters en ny stokastisk, agentbaserad modell som kännetecknas av (1) realistisk fysisk geografi av ön och andra miljömässiga begränsningar, (2) individuella beslutsprocesser av agenter, (3) icke-ergodicitet av agentens beteende och miljö och (4) randomiserade agent-miljöinteraktioner. 
Modellen används tillsammans med de bästa tillgängliga data för att bestämma rimliga rumsliga och temporära mönster av avskogning och andra socioekologiska egenskaper på Påskön före européers ankoms. Vidare identifieras några icke-triviala förbindelser mellan mikroskopiska beslut eller begränsningar (till exempel lokal inneslutning av agentens handlingar eller deras anpassningsstrategi till miljöförstöring) och makroskopiskt beteende hos systemet som inte lätt kan försummas i en diskussion om påsköns historia före europeisk kontakt.