% !TEX root=frame_thesis.tex

\section{Introduction}


First resource are trees, which were cut and used for food production (sweet sap is emerging from cut stems \citet{Mieth2015}), the production of manufactured goods like ropes or tools as well as canoes for fishing, firewood for cooking and cremation and, not to forget, the construction and transport of the Moai (e.g.\ \citet{Diamond2011}). Due to the slow regrowth of the palm tree (e.g. \citet{Brander1998}) this resource is close to non-renewable. Furthermore, the Polynesian rat, introduced by the first settlers at least stopped regrowth of the tree






Easter Island's suitability for agriculture has been subject to excessive debate not least due to the first very contrary observations made by Roggeveen in 1722 as an `outstandingly fruitful' and by Cook in 1774 as an extremely poor and often sterile island \citep{Bahn2017}. %page 112   % who perceived the island's agricultural potential completely contrary \citep{Cauwe2011}.
Nevertheless, from archaeological studies we know that the Easter Island people cultivated sweet potato, taro, yam, banans, sugar cane and other crops.
Since sweet potato appears to be the dominant staple crop \citep{Louwagie2006}, I focus on this single crop here.

Despite the progress so far, hypothesis  have produced very different results with strong implications on the reconstructed history of the Rapa Nui.