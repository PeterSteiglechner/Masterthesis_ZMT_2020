% !TEX root=frame_thesis.tex

\chapter{Introduction}


%First resource are trees, which were cut and used for food production (sweet sap is emerging from cut stems \citet{Mieth2015}), the production of manufactured goods like ropes or tools as well as canoes for fishing, firewood for cooking and cremation and, not to forget, the construction and transport of the Moai (e.g.\ \citet{Diamond2011}). Due to the slow regrowth of the palm tree (e.g. \citet{Brander1998}) this resource is close to non-renewable. Furthermore, the Polynesian rat, introduced by the first settlers at least stopped regrowth of the tree






%Easter Island's suitability for agriculture has been subject to excessive debate not least due to the first very contrary observations made by Roggeveen in 1722 as an `outstandingly fruitful' and by Cook in 1774 as an extremely poor and often sterile island \citep{Bahn2017}. %page 112   % who perceived the island's agricultural potential completely contrary \citep{Cauwe2011}.
%Nevertheless, from archaeological studies we know that the Easter Island people cultivated sweet potato, taro, yam, banans, sugar cane and other crops.
%Since sweet potato appears to be the dominant staple crop \citep{Louwagie2006}, I focus on this single crop here.

%Despite the progress so far, hypothesis  have produced very different results with strong implications on the reconstructed history of the Rapa Nui.

\section{The dispute about pre-historic Easter Island}

\begin{itemize}
	\item Easter Island's Pre-history is a fiercely debated topic
	\item Different theories exist on the population dynamics \citet{Hunt2007} and \citet{Diamond2011}/\citet{Bahn2017}.
	\item The differences occur because archaeological data is connected to large uncertainties (e.g.\ tree patterns \citet{Rull2020})
	\item Mathematical modelling has helped to explain possible scenarios of the dynamics and give interpretations of the historical developments.
	\item So far, all mathematical models focus on differential equation based approaches for the aggregate population of Easter Island.
	\item \citet{Brander1998}
	\item Variations of \citet{Brander1998}: Summarise the reviews in \citet{Reuveny2012}, \citet{Merico2017}.
	\item Social institution Extension of \citet{Brander1998}: \citet{Good2006}.
	\item Spatial Diffusion and Rats extension of \citet{Brander1998}: \citet{Basener2008}.
	\item The problem of these models is that with little variation in the parameters, any population dynamics can be achieved \citep{Brandt2015}.
\end{itemize}

\section{Agent Based Modelling of human resource interactions}
\begin{itemize}
\item What is Agent Based Modelling
\item The basic structure of an ABM in human resource interactions: Agents in environment, Update all agents asynchronously, update environment. Within each update agents change their local environment and their behaviour/properties are in turn influenced by the local environment. 
\item It breaks with ODE-type of modeling because it enables: 
\begin{itemize}
	\item spatially explicit modelling,
	\item  emergent global behaviour from local rules, 
	\item computational irreducibility of agent behaviour in terms of crisis. 
	\item Non-ergodic relation between agents and their environment.
	\item Stochasticity is natural in the decision making process, due to an agent's imperfect knowledge of the global sitution
\end{itemize}
\item ABM in ancient historical population dynamics: Maya \citep{Heckbert2013} and Anasazi \citep{Axtell2002}
\item Why ABM could help in Easter Island modelling: \citet{Merico2017}.
\item Structure of this thesis.
\end{itemize}

