% !TEX root=frame_thesis.tex

\chapter{Introduction}\label{chapter:Introduction}

\paragraph{Population Growth and the Malthusian Theory.}
Over the last century, human activity has become a key driver of not only local but also global environmental change. %, which is often denoted as the `Anthropocene'.
Given the limited resources of our planet, a global population growth from less than $2$ to current $7.8$ billion individuals within a century raises serious concerns.% about to the carrying capacity of the planet. %potential population overshoot and corresponding fatal consequences for the human society.
One of the first theories suggesting a scenario of overshooting population was developed by \citet{Malthus1798}. %, in which he draws a grim picture of a potential anthropogenic trajectory: As
He suggested that, as population size increases exponentially, resources such as fertile land exhaust, leading to increased prices, conflict and the general potential of a collapse of the human society.
Later and with a main focus on environmental degradation, the Club of Rome followed this rationale in their report on `The Limits of Growth' \citep{clubofrome1972}.
Of course, so far no global overshoot has occurred throughout history. 
It is now more commonly accepted that a global population follows (at maximum) a logistic rather than an exponential growth (e.g.\ in \citen{KC2017} (Figure 1)\footnote{However, this does not necessarily mean that resource and energy consumption also do not grow exponentially.}). 
Furthermore, technological advances, such as nitrogen fertilisation, have greatly enhanced the provision of necessary resources for humanity.
Today, however, the awareness about climate change as a real and imminent threat to the planet fuels interest in the Malthusian theory anew \citep{Reuveny2012}.
Hence, it is important to investigate the potential implications for the human society of scenarios of population overshoot.
%If only to be prudent, we must consider whether Malthus basically foresaw a real problem and miscalculated only its geographical scope and timing.




\paragraph{Examples of Overpopulation and Resource Exploitation in Ancient Civilisations.}
Interesting case studies of potential overpopulation often in connection to environmental degradation are provided by the disappearance of relatively small, ancient civilisations, such as the Maya in Meso-America, the Anasazi in the South West of the United States, or the Norse in Greenland \citep{Diamond2011}.
In particular, the fate of the Easter Island civilisation, the `Rapa Nui', has captured scientific and public interest for centuries.
One reason is the total isolation of the island for several centuries (see Figure \ref{fig:Karte}).
This provides an ideal, natural laboratory of a socio-ecological system in transformation. 
%, providing an ideal, closed laboratory case of a socio-ecological system.
Another reason is the substantial environmental change detected on the island following human settlement.  
Within a relatively short period of time, the initial dense palm tree forest turned into barren land deprived of any large vegetation as observed by the first European voyagers in the 18th century.
Finally and most importantly, the platforms (`Ahu') and hundreds of statues (`Moai') carved, transported to and erected in various places on the island indicate a large population with rich culture, advanced technological know-how, well established political institutions and a working trading economic system \citep{Cauwe2011}.
This impression is in stark contrast to the observed deforested island, with a relatively small and potentially even rivalrous civilisation reported especially in later visits to the island after $1770\, {\rm A.D.}$.
Resolving this mystery behind the Rapa Nui before European contact has arguably become one of the most intriguing questions in archaeology.


\begin{figure}
	\centering
	\includegraphics[width=1\linewidth]{images/easter_is_map_ago_v3_jpeg}
	\caption{Map of Easter Island today and its location in the Pacific taken from \citet{Merico2017}. Map courtesy of Jailson Fulgencio de Moura.}
	\label{fig:Karte}
\end{figure}

\paragraph{Facts about Easter Island History.}
Numerous studies in the fields of archaeology and anthropology have tried to reconstruct the history of the island and its people before European contact.
However, data is often sparse and can be interpreted in various ways \citep{Merico2017}.
For example, many Moai have been toppled in the 18th century which was interpreted as a sign of violence \citep{Bahn2017} or a religious/cultural burial process \citep{Cauwe2011}.
Often theories on the population dynamics are based on palynological data, i.e.\ pollen records from sediments in the three crater lakes and charcoal data from excavations.
However, there are substantial uncertainties in the dating of such data due to the process of sedimentation including geological gaps and drought periods (\citen{Hunt2007}; \citen{Bahn2017}), in the extrapolation to an island-wide data record \citep{Rull2020} and, most importantly, in the interpretation of these proxies for human activity and, thus, population dynamics (e.g.\ \citen{Cole2008}).
While many aspects of Easter Island history are debated, there are a few typically commonly accepted facts.
The relevant ones for this thesis include:
\begin{itemize}
	\item Before human settlement, the island was forested (\citen{Mieth2015}; \citen{Rull2010}).
	\item Polynesian rats, which were introduced to the island by the first settlers as a protein source, quickly reproduced and effectively hindered the regeneration of palm forests by feeding on the seeds of the palm trees, thus potentially reducing or even preventing tree regrowth (\citen{Hunt2007} and \citen{Bahn2017}).
	\item Starting from around $1200\,{\rm A.D.}$, there is evidence for intensified deforestation in some places on the island \citep{Rull2020} but with unknown contributions from different drivers (rats, climate or humans).
	\item The Rapa Nui farmed land and cultivated sweet potato, taro, yam, and many other crops. They used advanced methods, like lithic mulching, to increase yields (e.g.\ \citen{Louwagie2006}). However, the fertility of the soil and the climatic conditions for farming remain open questions \citep{Bahn2017}.
	\item The Moai were carved from the 13th to 17th century, followed by a societal change in the 17th or 18th century \citep{Cauwe2011}.
	\item European voyagers arrived in $1722\, {\rm A.D.}$ and more frequently after $1770\,$${\rm A.D.}$. They found a (nearly) treeless island and vaguely estimated the population size to be `thousands' in $1722\, {\rm A.D.}$, between $1000$ and $3000$ in the late 18th century and only $111$ in $1872\, {\rm A.D.}$ following devastating slave trade and a smallpox epidemic brought by Europeans \citep{Bahn2017}.
\end{itemize}

\paragraph{Main Uncertainties and Different Narratives of Easter Island History.}
The uncertainties in the dynamics of deforestation and population size have lead to two major, contrasting narratives about the natural and anthropogenic contributions to deforestation, the correlated population dynamics and, therefore, the overall sustainability of the Easter Island civilisation.
On the one hand, a `genocidal' view \citep{Hunt2007}, on the other hand, an `ecocidal' view (\citen{Diamond2011}; \citen{Bahn2017}).
According to the genocidal view, the first settlers arrived around $1200\,{\rm A.D.}$, the population size quickly grew to a plateau of $4000$ individuals and remained at this level for several centuries. Deforestation is attributed not only to human activity but was crucially fostered by the fast expansion of the Polynesian rats. 
However, the human society remained mostly resilient to this environmental degradation only to eventually be diminished by the introduction of European diseases and slave trade. 
According to the ecocidal view, the arrival of the first settlers occurred before $1000\, {\rm A.D.}$. 
The population then grew steadily and burnt trees to clear land for agriculture. This development intensified around $1200\, {\rm A.D.}$. 
Population size peaked some time after $1500\, {\rm A.D.}$ at levels estimated between $6000$ to $8000$ or $10000$ to $20000$ \citep{Bahn2017}, followed by a steep decline or even `collapse' \citep{Diamond2011} before European contact.
The narrative concludes that the Easter Island population overexploited the natural resources and, consequently, got locked in the Malthusian catastrophe of resource shortage, cultural disruption, conflict and consequent population decrease.
%page 218
Next to these two major themes, various other theories have been put forward.
E.g.\ \citet{Brandt2015} suggest with modelling work (described more in Chapter \ref{chapter:Theory}) that a scenario of quick population growth followed by a slow-demise is most consistent with data of charcoal remains. 
A model developed by \citet{Cole2008} suggests the possibility of multiple periods of growth and collapse throughout Easter Island history. 
The authors used a far-from-equilibrium approach in their population model.
However they do not state reasons or drivers for the suggested periods of population decline. 
Furthermore, \citet{Rull2016} reviews evidence for climatological rather than anthropogenic drivers of deforestation, arguing for a more holistic perspective.
All of these contrasting narratives, ultimately, are not based on direct evidence for population sizes but rely on uncertain proxy data. 
Therefore, interpretations and, thus, assumptions vary strongly and might be `spectacularly ill-posed' \citep{Merico2017} if new evidence is discovered.
The dynamics of population size and deforestation and its causes remain a fiercely debated topic as the different narratives strongly correlate with personal perceptions of the Rapa Nui and their response to an environmental crisis. 


%according to which the society was resilient in the face of a rat-induced deforestation but was eventually diminished due to the introduction of European diseases, and an ecocidal view, according to which humans  overexploited their dominant resources and found themselves faced with a declining population size \citep{Bahn2017} or even a collapse \citet{Diamond2011} with various socio-political consequences.
%Other theories include a slow-demise 
%Climate 

\paragraph{Mathematical Modelling of Socio-Economic Systems of Ancient Societies.}
Easter Island history has been studied in various disciplines, from archaeologists to anthropologists. 
A major contribution, however, has been made by socio-economic modelling.
Mathematical modelling, typically with macroscopic system models, has 
provided a workbench for testing hypotheses and scenarios of natural and anthropogenic dynamics on Easter Island given a certain set of assumptions.
I describe these models and the shortcomings they share in Chapter \ref{chapter:Theory}.
In this thesis, I take a different approach that is based on Agent-Based Modelling (ABM).
Such a new model can help to understand the complexity of the socio-ecological microcosm of the island by focusing on individual behaviour, uncertainty and spatial constraints rather than simply adding another macroscopic theory to the research body with assumptions based on insufficient data.
%implications and constraints to the spatial and temporal patterns of the civilisation and the environment in the face of microscopic, individual behaviour and uncertainty.

\paragraph{Outline}
In Chapter \ref{chapter:Theory}, I describe both types of mathematical models, the various existing macroscopic system models and the approach of Agent-Based Modelling considered in this thesis.
%I continue by pointing out shortcomings in the approaches used so far and explain the need for an Agent Based Model.
In Chapter \ref{chapter:Methods}, I develop such an Agent-Based Model from scratch by defining a spatially explicit environment, human agents with specific features and their various actions and interactions. 
I also justify the choices and assumptions made in this model with conclusions or data from the existing literature and plausible arguments, when data is not available.
In Chapter \ref{chapter:Results}, I then present some of the results obtained with this model and discuss the implications of different choices of assumptions and parameter values on the spatial and temporal dynamics produced by the model. 
% for the spatio-temporal history of Easter Island given variations of modules and parameters in the model.
