% !TEX root=frame_thesis.tex

\chapter{Introduction}\label{chapter:Introduction}

%First resource are trees, which were cut and used for food production (sweet sap is emerging from cut stems \citet{Mieth2015}), the production of manufactured goods like ropes or tools as well as canoes for fishing, firewood for cooking and cremation and, not to forget, the construction and transport of the Moai (e.g.\ \citet{Diamond2011}). Due to the slow regrowth of the palm tree (e.g. \citet{Brander1998}) this resource is close to non-renewable. Furthermore, the Polynesian rat, introduced by the first settlers at least stopped regrowth of the tree

%Easter Island's suitability for agriculture has been subject to excessive debate not least due to the first very contrary observations made by Roggeveen in 1722 as an `outstandingly fruitful' and by Cook in 1774 as an extremely poor and often sterile island \citep{Bahn2017}. %page 112   % who perceived the island's agricultural potential completely contrary \citep{Cauwe2011}.
%Nevertheless, from archaeological studies we know that the Easter Island people cultivated sweet potato, taro, yam, banans, sugar cane and other crops.
%Since sweet potato appears to be the dominant staple crop \citep{Louwagie2006}, I focus on this single crop here.

%Despite the progress so far, hypothesis  have produced very different results with strong implications on the reconstructed history of the Rapa Nui.

%\section{The Dispute about Pre-historic Easter Island}
%\subsection{Different Theories on the Population Dynamics}
%\begin{itemize}
%	\item Easter Island's Pre-history is a fiercely debated topic
%	\item Different theories exist on the population dynamics \citet{Hunt2007} and \citet{Diamond2011}/\citet{Bahn2017}.
%	\item The differences occur because archaeological data is connected to large uncertainties (e.g.\ tree patterns \citet{Rull2020})
%\end{itemize}
%Malthus
\paragraph{Population Growth and the Malthusian Theory}
Human activity in the recent century has left a pronounced mark on the natural environment such that the term `Anthropocene' is now widely accepted for the current geological epoch.
With exponentially rising concentration of greenhouse gases, plastic pollution, or biodiversity loss (among many other indicators) in a period often denoted as the `Great Acceleration', the human population is becoming a key factor in the environmental dynamics.
One of the most striking and foundational developments in this context is the growth of the global population from less than $2$ to currently $7.8$ billion within a century. 
Given the limited resources of our planet, this development rises questions about to the carrying capacity of the planet. %potential population overshoot and corresponding fatal consequences for the human society.
One of the first theories proposing an overshooting population was developed by Thomas Malthus.
In \citet{Malthus1798}, he draws a grim picture of the anthropogenic future: As the population size grows exponentially, resources such as fertile land exhaust,
leading to increased prices, conflict and the general potential of a collapse of the human society.
Later and with a main focus on environmental degradation, the Club of Rome followed this rationale in their report on `The Limits of Growth' \citep{clubofrome1972}.
Of course, so far no global overshoot has occurred throughout history, as technological advances such as Nitrogen fertilisation have greatly enhanced the planet's population carrying capacity.
Furthermore, it is now more commonly assumed that a global population follows a logistic rather than an exponential growth. 
However, as resources (and eventually also replacements for resources) are limited and the population dynamics and carrying capacity of a logistic growth cannot be easily determined, it is only sensible to study the implications for our human society in a scenario of population overshoot, especially with climate change posing a real and imminent threat to the overall livability on the planet.

%If only to be prudent, we must consider whether Malthus basically foresaw a real problem and miscalculated only its geographical scope and timing.


%Instead of trying to predict the dynamics of a global-scale population overshoot in our complex socio-economic system, 
\paragraph{Examples of Overpopulation and Resource Exploitation in Ancient Civilisations}
The disappearance of smaller, ancient civilisations, such as the Maya in Meso-America, the Anasazi in the South West of the United States, or the Norse in Greenland, often in connection to environmental degradation make for interesting case studies.
%\citet{Diamond2011} argues that ancient civilisations  the Maya, the Anasazi, the Norse in Greenland,...
In particular, the fate of the Easter Island civilisation, the `Rapa Nui', has captured scientific and public interest alike for centuries.
One reason is the total isolation of the remote, Pacific island over several centuries, providing an ideal, closed laboratory case of a socio-ecological system.
Another is the substantial environmental change detected on the island following human settlement.  
Within a relatively short period of time, the initial dense palm tree forest turned into barren land deprived of any large vegetation as observed by the first European voyages to the island in the 18th century.
Finally and most importantly, the platforms (`Ahu') and hundreds of infamous statues (`Moai') carved, transported to and erected in various places on the island indicate a large population with rich culture and technological know-how, well established political institutions and a working trading economic system.
This however, stands in stark contrast to the observed deforested island, with a relatively sparse and potentially even rivalrous civilisation reported especially in later visits to the island after $1770\, {\rm A.D.}$, immediately raising the question of the story behind the Rapa Nui.

%Finally the erection of platforms and hundreds of the infamous statues in various places on the island despite the apparent lack of large scale vegetation indicating a rich culture with extensive technological advances, well established 
%and at some point decimated population has posed an unsolved mystery ever since the first European witness reports.
\paragraph{Facts about Easter Island History}
Numerous studies in the fields of archaeology and anthropology have tried to reconstruct the history of the island and its people.
However, data is often sparse and can be interpreted in various ways \citep{Merico2017}.
E.g.\ many Moai have been toppled in the 18th century which was interpreted as a sign of violence \citep{Bahn2017} or a religious/cultural burial process \citet{Cauwe2011}.
Often theories on the population dynamics are based on pollen records and charcoal data from sediments in the three crater lakes.
However, there is a series of uncertainties when interpreting to such data such as the dating, the process of sedimentation including geological gaps and, most importantly, the implication for the population dynamics given these proxies (\citet{Bahn2017}, \citet{Hunt2007}, and \citep{Cole2008}).
There is little commonly accepted facts about Easter Island's history.
The relevant ones for this thesis include:
\begin{itemize}
	\item Before human settlement, the island was forested (\citet{Mieth2015}, \citet{Rull2010}).
	\item Rats were introduced to the island by the first settlers as a protein source, quickly reproduced and effectively hindered the regeneration of palm forests by gnawing on the seeds (\citet{Hunt2007} and \citet{Bahn2017})
	\item Starting from around $1200\,{\rm A.D.}$, there is evidence for intensified deforestation in some places on the island \citep{Rull2020} but unknown causes.
	\item The Rapa Nui farmed land and cultivated sweet potato, taro, yam and many other crops. They used advanced methods, like lithic mulching, to increase the harvest yields (e.g. \citep{Louwagie2006}). However, the suitability of the soil and climatic conditions for farming remain an open question \citep{Bahn2017}.
	\item The Moai were carved from the 13th to 17th century, followed by a societal change in the 17th or 18th century \citep{Cauwe2011}.
	\item European Voyages arrived in $1722\, {\rm A.D.}$ and more frequently after $1770\, {\rm A.D.}$. They found a (nearly) treeless island and vaguely estimated the population size to be `thousands' in $1722\, {\rm A.D.}$, between $1000$ and $3000$ in the late 18th century and only $111$ in $1872\, {\rm A.D.}$ following devastating slave trade and a smallpox epidemic brought by Europeans \citep{Bahn2017}.
\end{itemize}

\paragraph{Main Uncertainties and Different Narratives of Easter Island History}
The uncertainties in the dynamics of deforestation and population size have lead to two major, contrasting narratives about the natural and anthropogenic deforestation, the correlated population dynamics and, therefore, the overall view of sustainability of the Easter Island civilisation.
%In order to understand the succession of this diminishing culture, it is interesting to study the succession of events that lead to the extinction to the extinction of a highly 
%but also advanced and labour intensive methods used for cultivating plants on the island by the prehistoric society.
%When the first European voyages discovered the Pacific island in the 18th century, they found a society of a few thousand individuals 
%While we now know that the island must have been densely forested, at the time of discovery the island was nearly deprived of large vegetation.
%Nevertheless, several hundred platforms (`Ahu') and statues (`Moai') were carved, transported and erected in various places on the island.
%Advanced and labour intensive methods for cultivating plants.
%Since then, numerous studies in the fields of archaeology and anthropology have tried to reconstruct the history of the island and its people.
%Two main narratives Easter Island emerged:
On one hand a `genocidal' view \citep{Hunt2007}, on the other hand an `ecocidal' view (\citet{Diamond2011} and \citet{Bahn2017}).
According to the genocidal view, the first settlers arrived around $1200\,{\rm A.D.}$, the population size quickly grew to a plateau of $4,000$ individuals and remained at this level for several centuries. Deforestation is attributed to both human activity and the fast expansion of the Polynesian rats. 
However, the human society remained mostly resilient to this environmental degradation only to eventually be diminished by the introduction of European diseases and slave trade. 
According to the ecocidal view, the arrival of the first settlers occurred before $1000\, {\rm A.D.}$. 
The population then grew steadily and deforested and burnt trees to clear land for agriculture. This development intensified around $1200\, {\rm A.D.}$. 
Population size peaked some time after $1500\, {\rm A.D.}$ at levels estimated between $6,000$ to $8,000$ or $10,000$ to $20,000$ \citep{Bahn2017}, followed by a steep decline or even `collapse' \citep{Diamond2011}.
The narrative concludes that the Easter Island population overexploited the natural resources and, consequently, got locked in the Malthusian catastrophe of resource shortage, cultural disruption, conflict and consequent population decrease.
%page 218
Next to these two major themes, various other theories have been put forward.
\citet{Brandt2015} suggest a quick population growth followed by a slow-demise in their modelling study.
A model developed by \citet{Cole2008} suggests multiple periods of growth and collapse throughout history. 
Furthermore, \citet{Rull2016} reviews evidence for climatological impact rather than anthropogenic drivers of the deforestation, arguing for a more holistic perspective on Easter Island.
The dynamics of population size and deforestation and its causes remain a fiercly debated topic as the different narratives strongly correlate with our impression of the Rapa Nui in the face of an environmental crisis. 


%according to which the society was resilient in the face of a rat-induced deforestation but was eventually diminished due to the introduction of European diseases, and an ecocidal view, according to which humans  overexploited their dominant resources and found themselves faced with a declining population size \citep{Bahn2017} or even a collapse \citet{Diamond2011} with various socio-political consequences.
%Other theories include a slow-demise 
%Climate 

\paragraph{Mathematical Modelling of Socio-Economic Systems of Ancient Societies}
Easter Island history is studied in various disciplines, from archaeologists to anthropologists. 
A major contribution, however, has also been made by the field of socio-economic modelling of ancient societies.
Mathematical modelling, typically macroscopic system models, helped to reconstruct possible scenarios by calculating the impacts of certain assumptions ofnatural and anthropogenic processes on Easter Island.
I describe these models of Eater Island in more detail in Chapter \ref{chapter:Theory}, however they share the same insufficiencies.
In this thesis, therefore, I take a different approach of Mathematical Modelling for socio-economic modelling of Easter Island, an Agent Based Model, to understand the implications and constraints to the spatial and temporal patterns of the civilisation and the environment in the face of microscopic, individual behaviour and uncertainty.

\paragraph{Outline}
In Chapter \ref{chapter:Theory} I describe both types of mathematical models, the various existing, macroscopic system models and the new approach of Agent Based Modeling pursued in this thesis.
%I continue by pointing out shortcomings in the approaches used so far and explain the need for an Agent Based Model.
In Chapter \ref{chapter:Methods}, I develop such an Agent Based Model from scratch by defining a spatially explicit environment, agents with specific features and their various actions and interactions and justify the choices and assumptions made in this model with literature (and logical reasoning, where necessary).
Chapter \ref{chapter:Results}, then points out some results obtained with this model and discusses the implications for the spatial and temporal patterns of the history of Easter Island given the choice of parameters or processes.

\chapter{Mathematical Theory}\label{chapter:Theory}
\FloatBarrier
\section{Macroscopic, Differential Equation Based Approaches for Easter Island}
%\begin{itemize}
%	\item Mathematical modelling has helped to explain possible scenarios of the dynamics and give interpretations of the historical developments.
%	\item So far, all mathematical models focus on differential equation based approaches for the aggregate population of Easter Island.
%	\item \citet{Brander1998}
%	\item Variations of \citet{Brander1998}: Summarise the reviews in \citet{Reuveny2012}, \citet{Merico2017}.
%	\item Social institution Extension of \citet{Brander1998}: \citet{Good2006}.
%	\item Spatial Diffusion and Rats extension of \citet{Brander1998}: \citet{Basener2008}.
%	\item The problem of these models is that with little variation in the parameters, any population dynamics can be achieved \citep{Brandt2015}.
%\end{itemize}

\paragraph{The Basic Ordinary Differential Equation Model}
The theories about population dynamics and deforestation is often supported by a mathematical, human-resource interaction model.
By making assumptions about these interaction sometimes based on archaeological data, these models can produce realistic population dynamics and thereby explain or rule out certain scenarios under the used set of assumptions. 
So far, all mathematical models on Easter island base on a aggregate, predator-prey type of interactions between humans and resources provided by the environment (so called Lotka-Voltera model).
The dynamics of these macroscopic variables is then described by often non-linear, coupled ordinary differential equations.
The first and most famous model was developed by \citet{Brander1998}. 
They assumed two variables, an exponentially growing human population size $L$ (the predator) and a logistically growing primary, open-access resource with stock $S$ (the prey).
This resource can be interpreted as the forest/soil complex and is slowly renewable in the model and consumption of it enables population growth. 
The human population harvests $H(L,S)$ from the resource, thereby depleting the stock and increasing the fertility of the population.
The set of equations is
\begin{eqnarray}\label{eq:Brander}
\frac{dL}{dt} = L \cdot (b-d) + \phi \cdot H(L,S) \\
\frac{dS}{dt} = G(S)\cdot S - H(L,S) \ ,
\end{eqnarray}
where $b$ and $d$ are the constant birth and death rates, $\phi$ denotes the increase of fertility with resource consumption and $G(S)$ is the logistic growth rate of the resource.
This model reproduced a `boom and bust' cycle (Figure \ref{fig:brander1998eibasecase}), which as they argue is an example of the `Malthusian forces [leading] to a depletion of the resource base and social conflict' and thereby supports the ecocidal view of Easter Island's pre-history.
\begin{figure}
	\centering
	\includegraphics[width=1 \textwidth]{images/Brander1998_EIBaseCase}
	\caption{Replication of the model in \citet{Brander1998} with parameter setting as in the `Easter Island Base Case' in Figure 3 of the paper. The dots on the right represent the equilibrium values.}
	\label{fig:brander1998eibasecase}
\end{figure}

\paragraph{Extensions of the Basic Model}
In the last two decades several extensions and adjustments have been added to the first mathematical model by \citet{Brander1998}.
\citet{Reuveny2012} provides an extensive overview of these models.
\citet{Merico2017} summarises gaps and advances within this field of research. 
Here, I point out a few variations of the model which are especially relevant to the addition of this thesis.
\citet{dAlessandro2007} separates the resource dependency from one single stock to two, an inexhaustible and a renewable resource, that tends to irreversibly exhaust if below a certain threshold.
The author finds multiple stable states due to this disaggregation of the ecological variable. 
A model by \citet{Good2006} introduces foresight and resource management institutions (either a system of property rights or a social planner) to the simple predator-prey model by \citet{Brander1998}.
Here, the harvest rate is determined by maximising a utility function over a certain time horizon with reasonable discounting.
However, the authors find that even with optimal institutions in place, collapse is inevitable.
\citet{Basener2008} extended the model by another variable accounting for the rat population and their devastating impact on tree regrowth as previously found by \citet{Hunt2007}. 
This model was then further extended \citep{Basener2011} to include a spatial component. 
A one-dimensional, discrete space defined and diffusion of rats and population allowed between adjacent cells. 
The authors find that simply changing the mobility of rats can qualitatively alter the dynamics of the population size.
To date, the overly simplified spatial representation is the only model including a specific spatial representation of the population dynamics on Easter Island to my knowledge.
Finally, in a recent analysis, \citet{Brandt2015} developed a model of a system of ODEs for human population, rats and trees covering both the human resource interaction (based on \citet{Basener2008}) as well as a disease spreading model.
They found that through variation of only a few parameters in a reasonable range, any of the proposed narratives (ecocide, genocide or the slow-demise) can be obtained.
The extensive body of research presented by these models has given rise to evaluate the consequences. 

\paragraph{Shortcomings of Macroscopic ODE Models}
Despite the extensive research on modelling the Easter Island history with macroscopic system models, the dispute between different narratives could not be put aside.
The ODE based models, the main tool for Easter Island modelling, typically either show the same (inherent) boom and bust cycle or showed shortcomings in that different choices of parametrisation within the uncertainties in archaeological data lead to contrasting results and narratives consistent with the sparse and uncertain data.
%ODE-type modelling can only have limited significance.
Many shortcomings of these models are inherent problems of macroscopic system modelling, e.g.\ the missing spatial constraints, heterogeneity, co-evolving behaviour and emergent phenomena.
A microscopic, agent-based model therefore builds a useful supplement to these macroscopic approaches in addressing many of the problems as described in the next Section.
%E.g.\ the model by \citet{Brander1998} assumes open access to the resources which defers intuition since the location of an individual restricts the availability of resources. 
%Other models \citep{Good2006} might include a more complex economic which incorporates restrictions on resource harvest but implemented via a social planner.
%Thus, such dynamics are not co-evolving or emergent but externally established 

%While models like \citet{Good2006} have restrictions on the amount of resources, it require a social planner and it is not so much an emergent restriction of the population dyanmics but rather a institution.

\FloatBarrier
\section{Agent Based Modelling of human resource interactions -- Overview, Motivation and Previous Approaches}
%\begin{itemize}
%\item What is Agent Based Modelling
%\item The basic structure of an ABM in human resource interactions: Agents in environment, Update all agents asynchronously, update environment. Within each update agents change their local environment and their behaviour/properties are in turn influenced by the local environment. 
%\item It breaks with ODE-type of modeling because it enables: 
%\begin{itemize}
%	\item spatially explicit modelling,
%	\item  emergent global behaviour from local rules, 
%	\item computational irreducibility of agent behaviour in terms of crisis. 
%	\item Non-ergodic relation between agents and their environment.
%	\item Stochasticity is natural in the decision making process, due to an agent's imperfect knowledge of the global sitution
%\end{itemize}
%\item ABM in ancient historical population dynamics: Maya \citep{Heckbert2013} and Anasazi \citep{Axtell2002}
%\item Why ABM could help in Easter Island modelling: \citet{Merico2017}.
%\item Structure of this thesis.
%\end{itemize}

\paragraph{Overview of Agent Based Modeling}
In Agent based models (ABMs) a system is grown bottom-up from its constituent units.
The perspective of an ABM therefore focuses on the microscopic rather than macroscopic system.
Originating from the field of Artificial Intelligence in the recent past and enabled by the significant increase in computational power, ABMs are now commonly used tools at the centre between cognitive psychology, game theory and complexity science \citep{Bousquet2004}.
An ABM simulates a number of agents, often situated in an environment, with specific traits.
Agents (and the environment) are updated asynchronously at certain timesteps over the simulated period.
In each update, a single agent interacts with the environment and other agents according to a set of individual rules.
These rules are usually heterogeneous, non-linear (e.g.\ discontinuous or discrete), stochastic, time-dependent and adaptive and might depend on memory and path-dependency \citep{Bonabeau2002}.
In the case of a spatial model, rules and behaviour additionally depend on the explicit location of an agent.
Usually, agents make individual decisions, based on perceptions of their local surroundings and their internal state, and thereafter act independently and autonomously.
With this microscopic setup, overall macroscopic dynamics of the system are obtained.
Aggregate system variables can then be interpreted both as outcomes of and as contexts for the agents' decisions and actions \citep{Kohler2000}.

%The field of Agent Based Modelling originally comes from Artificial Intelligence study. 
% WHERE FROM 
% INFORMATICS, EXAMPLES, ...


%These are updated asynchronously in certain timesteps over the simulation period. 
%In each update, a single agent interacts with the environment and other agents and adapts its own features.
%Usually, agents act independently and based on individual decision making by evaluating their state at the current time.
%Aggregate variables of all agents are then a combination of context as well as outcome \citep{Kohler2000}.



\paragraph{Advantages with respect to ODE Models}
In many complex systems, ABMs are advantageous over macroscopic system models by accounting for the following system properties \citep{Bookstaber2019}:
\begin{itemize}
	\item Emergent phenomena from local, heterogeneous behaviour of the agents,
	\item Computational irreducibility of agent behaviour (e.g.\ in the decision making in times of crisis),
	\item Non-ergodicity of relations between agents and their environment (i.e.\ conditions and rules of behaviour co-evolve with the agent and environment \citep{Kohler2000}),
	\item Stochasticity in the actions and decision making processes, due to imperfect knowledge (or uncertainty) of the agents. 
\end{itemize}
Additionally, ABMs have the advantage of allowing for a very natural and flexible implementation of an explicit space dependency of rules and a spatially heterogeneous environment
In ODE models, space dependency is typically implemented via non-intuitive, complex diffusion processes (compare to the model by \citet{Basener2011}).
%In such systems, ABMs are often the most natural and flexible way of describing a system.

%An ABM is especially relevant if a system exhibits emergent phenomena, i.e.\ individual behaviour according to rules cummulates in complex macroscopic patterns.
In particular if a system exhibits emergent phenomena, reducing the heterogeneity of agents to one representative agent by averaging as done in macroscopic ODE models, crucial information is lost.
Instead, an ABM generates emergence from bottom-up by accounting for non-linear feedbacks due to heterogeneous agents and stochasticity e.g.\ connected to uncertainty in the decision making process.

%Furthermore stochastic, discrete processes as in an ABM lead to complex responses of the model outcome, that can not be simply 
% while an ABM gives rise to non-linear feedbacks of fluctuations e.g.\ in the case of uncertainty.
%In such cases, simple averaging and expressing processes by a representative behaviour, as applied in macroscopic system modelling, does not capture emergent phenomena.
%However, an ABM gives rise to non-linear feedbacks from heterogeneous, fluctuating agents e.g.\ in the face of uncertainty and imperfect knowledge in decision making.


\paragraph{Agents Based Modelling in Socio-Ecological Systems}
ABMs have traditionally been applied to problems connected to flows, markets, organisations, or diffusion \citep{Bonabeau2002}.
Typical applications include e.g.\ traffic jams, ant streets or swarm behaviour of fish and bird flocks.
However, ABMs are also a common approach for socio-ecological systems \citep{Muller-Hansen2017}.
Such a system comprises complex, co-evolution of humans and the environment, which interact with each other in complex, non-linear, adaptive interactions on multiple time and spatial scales.
This co-evolution, heterogeneity of agents and explicit spatial dependency make socio-ecological systems suitable application cases of ABMs \citep{Bousquet2004}.

%The population dynamics of ancient societies 
%, as agents in such a system usually also act heterogeneously and dependent on local space.
%Here, social structure such as the emergence of political organisations in the agent-agent interactions are at the heart of Agent-Based models.
%As agents in ecological systems usually act heterogeneously and strongly dependent on their local environment or neighbourhood, the usage of ABM with an explicit space dependency is suitable \citet{Bousquet2004}.
% motivates the use of ABM in ecological modeling with the local response 

%According to \citet{Bousquet} ABMs in the ecosystem context can be applied to either create scenarios showing `what might be rather than what is' to understand a system or to produce reality-like scenarios in order to test `what if' questions.

%The disadvantage of ABM is that there's no mathematical proof.Bousquet2004.

\paragraph{ABMs for Ancient Civilisations}
Agent Based modelling has been applied to study the history of two ancient civilisations.
Firstly, \citet{Axtell2002} (and \citet{Janssen2009}), use an ABM to reproduce the spatio-temporal history of the Anasazi society in Arizona, US and its disappearance around $1300\, {\rm A.D.}$. 
Heterogeneous agents are various attributes represent households that interact with the environment via farming. 
Agent-Environment interaction bases on a set of anthropologically plausible rules and the environment is determined from an extensive data record of harvest yield potential over time. 
Agents choose a location for a farm depending on the maximum potential yield in a certain distance to water and for the nearest possible settlement with access to water. The harvest success determines the fertility and thus the agent and overall population dynamics.
Secondly, \citet{Heckbert2013} develops an ABM of the Maya civilisation resulting in a `somewhat analogous' reproduction of the spatial pattern and a timeline.
The agents in this model generate a certain amount of agricultural yield on cells of a discretised map based on a benefit-cost assessment by the agent. 
They are further connected in clustered, adaptive trade networks, from which they benefit.
Both of these approaches attempt to explain the spatio-temporal history of an ancient civilisation by growing an Agent Based Model with anthropological rules and a biologically, geographically explicit environment on a discretised map.


\paragraph{Motivation for applying an ABM on Easter Island}
This thesis presents a similar Agent Based modelling approach for the history of Easter Island.
I have discussed many aspects and advantages of Agent Based Modelling which apply in socio-ecological systems, and for modelling ancient societies and Easter Island in particular.
Similar to the valley of the Anasazi, Easter Island is a small, confined space with distinct geographical and biological features with heterogeneous agricultural suitability.
\citet{Merico2017} further argues for the use of an ABM for Easter Island to overcome the shortcomings of ODE models and limits of the available archaeological data.
The main features of the ABM presented here include a spatially explicit environment, locally confined agent-environment interaction, a simple adaption strategy of agents to environmental degradation and individual, stochastic moving decisions by agents.

